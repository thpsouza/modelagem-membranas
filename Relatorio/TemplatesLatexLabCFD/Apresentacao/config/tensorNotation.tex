% Operações Matemáticas
% vector operations format
\newcommand{\gvec}[1] {\ensuremath{\mbox{\boldmath$\bf#1$}}}
\newcommand{\idot}[2] {#1 \cdot #2}
\newcommand{\rot}[2] {#1 \times #2}
\newcommand{\uv}[1]{\ensuremath{\mathbf{\hat{#1}}}} % for unit vector

% derivatives formats
\let\underdot=\d % rename builtin command \d{} to \underdot{}
% simple derivative
\renewcommand{\d}[2] {\frac{d #1}{d #2}}
\newcommand{\dn}[3] {\frac{d^{#3} #1}{d {#2}^{#3}}}
\newcommand{\D}[2] {\frac{D #1}{D #2}}
% partial derivative
\newcommand{\pd}[2]{\frac{\partial #1}{\partial #2}} 
\newcommand{\pdn}[3]{\frac{{\partial}^{#3} #1}{\partial {#2}^{#3}}}

% integral formats
% volume integrals
\newcommand{\intv}[2]{\int_{#1} #2 \,\mathrm{d}#1} 
\newcommand{\intg}[3]{\int_{#1_{#2}} #3 \,\mathrm{d}#1}

% surface integrals
\newcommand{\oints}[2]{\oint_{#1} #2 \,\mathrm{d}#1} 

% generic operations
\newcommand{\abs}[1]{\left| #1 \right|} % for absolute value
\newcommand{\avg}[1]{\left< #1 \right>} % for average

% tensor operations formats
\newcommand{\grad}[1]{\gvec{\nabla} #1} % for gradient
\let\divsymb=\div % rename builtin command \div to \divsymb
\renewcommand{\div}[1]{\gvec{\nabla} \cdot \left( #1 \right)} % for divergence
\newcommand{\curl}[1]{\gvec{\nabla} \times ( #1 )} % for curl
\newcommand{\laplacian}[1]{\gvec{\nabla}^{2} #1 } % for curl
\newcommand{\grads}[1]{\gvec{\nabla}_s #1} % for bidimensional gradient
\newcommand{\divs}[1]{\gvec{\nabla}_s \cdot \left( #1 \right)} % for bidimensional divergence

% subscripts & superscripts
\newcommand{\lam} {\text{lam}}           % laminar
\newcommand{\turb} {\text{turb}}         % turbulent
\newcommand{\A} {A}                      % chemical specie index

% variables
\newcommand{\U} {\mathbf{u}}             % velocity
\newcommand{\g} {\mathbf{g}}             % gravity
\newcommand{\n} {\uv{n}}                 % normal unit vector
\newcommand{\stress} {\gvec{\tau}}       % stress tensor
\newcommand{\viscous} {\stress^{\prime}} % viscous tensor

\newcommand{\rhs}{\texttt{r.h.s.}}

\newcommand{\CO}{CO\textsubscript{2}}
\newcommand{\CH}{CH\textsubscript{4}}
