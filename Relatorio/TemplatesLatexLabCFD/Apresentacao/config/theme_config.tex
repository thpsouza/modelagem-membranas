% Fonts and AMS math 
\usepackage{amsfonts}
\usepackage{amsmath}
\usepackage{amsthm}
\usepackage{amssymb}

% Movies, animations and more
\usepackage{multimedia}
\usepackage{keyval}
\usepackage{graphicx}
\usepackage{tikz}
\usepackage{color,xcolor}
\usepackage{epsfig}
\DeclareGraphicsExtensions{.pdf,.png,.gif,.jpg,.eps,.eps.gz}
\usepackage{textcomp}
\usepackage{hyperref}
\usepackage[super]{nth}
\usepackage{transparent}

\definecolor{lblue}{RGB}{68,69,185}
\definecolor{tungs}{RGB}{51,51,51}
\usepackage{tikz}
\usetikzlibrary{shadows,calc}

% code adapted from http://tex.stackexchange.com/a/11483/3954

% some parameters for customization
\def\shadowshift{3pt,-3pt}
\def\shadowradius{6pt}

\colorlet{innercolor}{black!60}
\colorlet{outercolor}{gray!05}

% this draws a shadow under a rectangle node
\newcommand\drawshadow[1]{
    \begin{pgfonlayer}{shadow}
        \shade[outercolor,inner color=innercolor,outer color=outercolor] ($(#1.south west)+(\shadowshift)+(\shadowradius/2,\shadowradius/2)$) circle (\shadowradius);
        \shade[outercolor,inner color=innercolor,outer color=outercolor] ($(#1.north west)+(\shadowshift)+(\shadowradius/2,-\shadowradius/2)$) circle (\shadowradius);
        \shade[outercolor,inner color=innercolor,outer color=outercolor] ($(#1.south east)+(\shadowshift)+(-\shadowradius/2,\shadowradius/2)$) circle (\shadowradius);
        \shade[outercolor,inner color=innercolor,outer color=outercolor] ($(#1.north east)+(\shadowshift)+(-\shadowradius/2,-\shadowradius/2)$) circle (\shadowradius);
        \shade[top color=innercolor,bottom color=outercolor] ($(#1.south west)+(\shadowshift)+(\shadowradius/2,-\shadowradius/2)$) rectangle ($(#1.south east)+(\shadowshift)+(-\shadowradius/2,\shadowradius/2)$);
        \shade[left color=innercolor,right color=outercolor] ($(#1.south east)+(\shadowshift)+(-\shadowradius/2,\shadowradius/2)$) rectangle ($(#1.north east)+(\shadowshift)+(\shadowradius/2,-\shadowradius/2)$);
        \shade[bottom color=innercolor,top color=outercolor] ($(#1.north west)+(\shadowshift)+(\shadowradius/2,-\shadowradius/2)$) rectangle ($(#1.north east)+(\shadowshift)+(-\shadowradius/2,\shadowradius/2)$);
        \shade[outercolor,right color=innercolor,left color=outercolor] ($(#1.south west)+(\shadowshift)+(-\shadowradius/2,\shadowradius/2)$) rectangle ($(#1.north west)+(\shadowshift)+(\shadowradius/2,-\shadowradius/2)$);
        \filldraw ($(#1.south west)+(\shadowshift)+(\shadowradius/2,\shadowradius/2)$) rectangle ($(#1.north east)+(\shadowshift)-(\shadowradius/2,\shadowradius/2)$);
    \end{pgfonlayer}
}

% create a shadow layer, so that we don't need to worry about overdrawing other things
\pgfdeclarelayer{shadow} 
\pgfsetlayers{shadow,main}

\newsavebox\mybox
\newlength\mylen

\newcommand\shadowimage[2][]{%
\setbox0=\hbox{\includegraphics[#1]{#2}}
\setlength\mylen{\wd0}
\ifnum\mylen<\ht0
\setlength\mylen{\ht0}
\fi
\divide \mylen by 120
\def\shadowshift{\mylen,-\mylen}
\def\shadowradius{\the\dimexpr\mylen+\mylen+\mylen\relax}
\begin{tikzpicture}
\node[anchor=south west,inner sep=0] (image) at (0,0) {\includegraphics[#1]{#2}};
\drawshadow{image}
\end{tikzpicture}}

% Header and footer
%\usepackage{fancyhdr}
\usepackage{etoolbox}
%\pagestyle{fancy}
%\fancyhf{}
%
%\makeatletter
%\patchcmd{\@fancyhead}{\rlap}{\color{lblue}\rlap}{}{}
%\patchcmd{\headrule}{\hrule}{\color{lblue}\hrule}{}{}
%\patchcmd{\@fancyfoot}{\rlap}{\color{lblue}\rlap}{}{}
%\patchcmd{\footrule}{\hrule}{\color{lblue}\hrule}{}{}
%\makeatother
%
% Header
%\renewcommand{\headrulewidth}{2.5pt}
%\lhead{UNIVERSIDADE FEDERAL DO RIO DE JANEIRO}
%\rhead{ESCOLA DE QUÍMICA}
%
% Footer
%\renewcommand{\footrulewidth}{0.25pt}
%\lfoot{EQE-044 Computação Científica Aplicada}
%\rfoot{\thepage}

% Themes and good-looking stuff
%\usetheme{Hannover}
\usetheme{Pittsburgh}
%\usetheme{metropolis}
%\usetheme{Boadilla}
\usecolortheme{dolphin}
\usefonttheme{professionalfonts}
\usefonttheme[onlymath]{serif}
\setbeamertemplate{footline}[frame number]
\setbeamertemplate{frametitle continuation}{}
%\setbeamertemplate{frametitle continuation}[from second]
% Take off the "boring" navigation buttons
\setbeamertemplate{navigation symbols}{}

\definecolor{beamer@Orange}{RGB}{233, 137, 59 }
\definecolor{beamer@Blue}{RGB}{49, 83, 161 }
\colorlet{beamer@BlueOpacity}{beamer@Blue!35}

%New hline length
\setlength{\arrayrulewidth}{1.5pt}

%Footer
\setbeamercolor{title in head/foot}{fg=white , bg=beamer@Orange}
\setbeamercolor{author in head/foot}{fg=white, bg=beamer@Blue}
\setbeamercolor{subsection in head/foot}{fg=white, bg=beamer@Blue}

%Table of Contents
\setbeamercolor{section number projected}{bg=beamer@Blue}
\setbeamercolor{section in toc}{fg=beamer@Blue}
\setbeamercolor{subsection in toc}{fg=black}

%Frame Title
\setbeamercolor{frametitle}{fg=beamer@Blue}

%Section Title
\setbeamercolor{sectiontitle}{fg=beamer@Blue}
\setbeamerfont{sectiontitle}{size=\Huge}

%Bullets
\setbeamercolor{itemize item}{fg=beamer@Blue}

%Frame to begin a section
\AtBeginSection[]{
  \begin{frame}
  \vfill
  \centering
  \begin{beamercolorbox}[sep=8pt,wd=\textwidth,ht=2.25ex,dp=1ex,left,shadow=true,rounded=true]{sectiontitle}
    \hskip 2.ex \usebeamerfont{sectiontitle}\insertsectionhead\par%
    \vskip-2.ex%
    \color{beamer@Orange}\rule{0.9\textwidth}{1.pt}
  \end{beamercolorbox}
  \vfill
  \end{frame}
}

%Title Frame
\makeatletter
\setbeamertemplate{title page}{
  \vbox{}
  \vfill
  \begingroup
    \centering
    \vskip-7.ex%
    \begin{beamercolorbox}[sep=8pt,center]{title}
      \usebeamerfont{title}\inserttitle\par%
      \ifx\insertsubtitle\@empty%
      \else%
        \vskip0.25em%
        {\usebeamerfont{subtitle}\usebeamercolor[fg]{subtitle}\insertsubtitle\par}%
      \fi%
    \end{beamercolorbox}%
    \vskip1em\par
    \begin{beamercolorbox}[sep=8pt,center]{author}
      \usebeamerfont{author}\insertauthor
    \end{beamercolorbox}
    \vskip-2.ex%
    % ----------------------- new
    \begin{beamercolorbox}[sep=8pt,center]{date}
      \usebeamerfont{date}\insertdate
    \end{beamercolorbox}\vskip0.5em
    {\usebeamercolor[fg]{titlegraphic}\inserttitlegraphic\par}
  \endgroup
  \vfill
}
\makeatother


%Footline for all frames
\makeatother
\setbeamertemplate{footline}
{
  \leavevmode%
  \hbox{%
  \begin{beamercolorbox}[wd=.37\textwidth,ht=2.25ex,dp=1ex,center]{author in head/foot}%
    \usebeamerfont{author in head/foot}\tiny{\insertshortauthor}%\hspace*{0.15cm}(\insertshortinstitute)
  \end{beamercolorbox}%
  \begin{beamercolorbox}[wd=.47\textwidth,ht=2.25ex,dp=1ex,center]{title in head/foot}%
    \usebeamerfont{title in head/foot}\tiny{\insertshorttitle}\hfill
  \end{beamercolorbox}%
  \begin{beamercolorbox}[wd=.16\textwidth,ht=2.25ex,dp=1ex,center]{subsection in head/foot}%
	\usebeamerfont{date in head/foot}    
    \insertframenumber{} / \inserttotalframenumber\hspace*{1ex}
  \end{beamercolorbox}}%
  \vskip0pt%
}

%Put logo in header
\addtobeamertemplate{frametitle}{}{
\begin{tikzpicture}[remember picture,overlay]
\node[anchor=north west] at (current page.north west) {\includegraphics[height=1.cm]{config/fig/eqLabCFD}};
\end{tikzpicture}
}

%\setbeamertemplate{part page}{
%        \begin{beamercolorbox}[sep=8pt,center,wd=\textwidth]{part title}
%            \usebeamerfont{part title}\insertpart\par
%        \end{beamercolorbox}
%}
% Listings package
\usepackage{listings}
\usepackage{caption}
\usepackage{courier}
\lstset{
         language=bash,
         basicstyle=\scriptsize\ttfamily,
         numbers=none,
%         numberstyle=\tiny,
%         stepnumber=3,
%         numbersep=5pt,
         tabsize=4,
         extendedchars=true,
         breaklines=true,
         keywordstyle=\color[rgb]{0,0,1}\textbf,
         identifierstyle=\color{black},
         commentstyle=\color[rgb]{0.1,0.5,0},
         stringstyle=\color[rgb]{0.9,0,1},
         numberstyle=\color[rgb]{0.205, 0.142, 0.73},
         frame=single,
         showspaces=false,
         showtabs=false,
         xleftmargin=13pt,
         framexleftmargin=13pt,
         framexrightmargin=3pt,
         framexbottommargin=2pt,
%         backgroundcolor=\color{lightgray},
         backgroundcolor=\color{white},
         showstringspaces=false,
%         title=\lstname                   % show the filename of files included with \lstinputlisting; also try caption instead of title
 }
 
\lstset{literate=
  {á}{{\'a}}1 {é}{{\'e}}1 {í}{{\'i}}1 {ó}{{\'o}}1 {ú}{{\'u}}1
  {Á}{{\'A}}1 {É}{{\'E}}1 {Í}{{\'I}}1 {Ó}{{\'O}}1 {Ú}{{\'U}}1
  {à}{{\`a}}1 {è}{{\`e}}1 {ì}{{\`i}}1 {ò}{{\`o}}1 {ù}{{\`u}}1
  {À}{{\`A}}1 {È}{{\'E}}1 {Ì}{{\`I}}1 {Ò}{{\`O}}1 {Ù}{{\`U}}1
  {ä}{{\"a}}1 {ë}{{\"e}}1 {ï}{{\"i}}1 {ö}{{\"o}}1 {ü}{{\"u}}1
  {Ä}{{\"A}}1 {Ë}{{\"E}}1 {Ï}{{\"I}}1 {Ö}{{\"O}}1 {Ü}{{\"U}}1
  {â}{{\^a}}1 {ê}{{\^e}}1 {î}{{\^i}}1 {ô}{{\^o}}1 {û}{{\^u}}1
  {Â}{{\^A}}1 {Ê}{{\^E}}1 {Î}{{\^I}}1 {Ô}{{\^O}}1 {Û}{{\^U}}1
  {œ}{{\oe}}1 {Œ}{{\OE}}1 {æ}{{\ae}}1 {Æ}{{\AE}}1 {ß}{{\ss}}1
  {ç}{{\c c}}1 {Ç}{{\c C}}1 {ø}{{\o}}1 {å}{{\r a}}1 {Å}{{\r A}}1
  {€}{{\EUR}}1 {£}{{\pounds}}1
}

\newcommand{\nologo}{\setbeamertemplate{logo}{}}
\newcommand{\nofootline}{\setbeamertemplate{footline}{}}


%\DeclareCaptionFont{white}{\color{white}}
%\DeclareCaptionFormat{listing}{\colorbox{lblue}{\parbox{\textwidth}{#1#2#3}}}
%\captionsetup[lstlisting]{format=listing,labelfont=white,textfont=white}
