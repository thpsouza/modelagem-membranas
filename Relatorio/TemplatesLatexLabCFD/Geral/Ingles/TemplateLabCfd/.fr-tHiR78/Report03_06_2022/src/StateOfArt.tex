\section{State of the art}
\label{sec:StateOfArt}
\subsection{SPS technologies in \Floco}
\citet{Oathkeeper2017} presents an extensive bibliographic review about SPM \Floco archetypes, SPS technologies and fields where they are used. In summary, the most important system listed in \citet{Oathkeeper2017} are: multiphase~\citep{Hua2012}, hybrid~\citep{Solvik2013} and centrifugal pumps to boost fluid stream to host facility; two-phase separators (gas + liquid)~\citep{Jahnsen2011}; three-phase separators (gas + water + oil)~\citep{Jahnsen2011}; centrifugal pump to water and seawater injection~\citep{Marjohan2014}; dry gas compressor to take gas stream to host facility~\citep{Storstenvik2016}. 

\subsection{Wet Gas Compression}
The wet gas compression is an alternative to boost unprocessed wellstream to host facility. Compared to dry gas compression, it does not require upstream scrubbers to remove liquid.~\citep{Storstenvik2016} It is different from  multiphase pumps boost due to higher volumetric throughput and power requirements~\citep{Boe2018}.

Usually, the wet gas compressor has to be prepared to deal with  1-5\% of liquid by volume, what can represent up to 50\% in mass~\citep{Boe2018}. Furthermore, the compressor has to operate with different multiphase flow regimes, consequence of pressure and liquid load changes, water and various forms o solids such as scale, asphaltenes, wax, hydrates and sand~\citep{Boe2018}.

Some wet gas compression advantages are kick-off dead wells~\citep{Boe2018}, unlock abandoned liquid reserves~\citep{Floisand2019} and regulate wells backpressure~\citep{Boe2018, Floisand2019}. Moreover, besides gas production, wet gas compression allows a cyclic production of more than 5,000 bbl/d~\citep{Floisand2019}. 

\citet{SubseaSurvey2018} report the subsea wet gas compression used at Gulfaks South Brent and \citet{SubseaSurvey2020} report a phase 2 conceptual project for the same field.

The operation of the first wet subsea compressor, WGC4000, occured without any flaw from 2017 until 2019~\citep{Floisand2019} after it had to be taken out of operation with one month of operation due to umbilical leakage~\citep{Boe2018}.  After its sucess, the next generation of wet gas compressors (WGC6000) has been tested to achieve compression requirements for major subsea gas fields~\citep{Floisand2019} such as significant improvements in volumetric capacity, differential pressure, and power rating~\citep{Floisand2019}. However, it requires attention in the power transmission and distribution system since it can demand as much power as 100 MW, corresponding to 50,000 homes~\citep{Floisand2019}.  

OneSubsea advertises a wet gas compressor that can deal from 0 to 100\% volume fraction, up to 689 bar, 3000 m of water depth, 120 km of distance to host facility~\citep{OneSubseaWetCompressor}. It also reports 10,000 consecutive hours of operation with 100\% availabilty, lower energy consumption and \CO~ emissions~\citep{OneSubseaWetCompressor}.  

\subsection{\CO~dense phase separation}
\label{sec:DensePhaseSeparation}
The Libra Block (from 1,700 to 2,400 at Santos Basin) reservoir fluid conditions, GOR of 440 Sm\textsuperscript{3}/Sm\textsuperscript{3} and \CO content in the gas varying from 40 to 45\% in volume~\citep{Melo2019}, became important to devise a \CO~subsea separation equipment which allows its separation and reinjection without arriving at topside. As \citet{Melo2019} stress out, Libra Consortium is still seeking for technological developments to overcome industry subsea offshore challenges to enhance the results of the up to four mega projects that will be installed in Mero field.

In this context, \HiSep~\citep{Passarelli2017} arises as an alternative. It is a subsea high-\CO-content gas dense phase separation at high pressures~\citep{Melo2019, Passarelli2019}. The high \CO~content in a hydrocarbon mixture causes a significant change in the phase envelope, generating at least a two-phase region: one hydrocarbon-rich phase and another one rich in \CO~\citep{Passarelli2019}.

\HiSep~can be used in Mero's reservoir where the produced gas exhibits a gas-like viscosity and a liquid-like density, when submitted to high pressures, which allows the use of gravity
separation and centrifugal pumps to directly reinject this rich \CO~gas stream into the reservoir, directly from the seabed~\citep{Melo2019}. As a consequence, it reduces significantly the volume of gas released at the topside, at low pressures~\citep{Melo2019}, which entails in the extension of the oil production plateau or the upgrade of the FPSO total oil production capacity whether it is applied since the concept development of the field~\citep{Melo2019}. Moreover, \HiSep~has the potential to reduce up to 50\% of the power required to reinject the high \CO~content gas and cut carbon footprint and \CO~emissions~\citep{Melo2019}.

\citet{deSouza2019} have described \HiSep~separation system as a depressurization valve, heat integration and heat exchanger upstream \citet{Hannisdal2012}'s compact vertical vessel with cyclonic devices improving separation performance as it seems to be \citet{Passarelli2017}'s contribution. Downstream the gravity separator, the \CO~poor phase goes to the FPSO and athe rich one passes to a cooler, injection pump and heat integration before the injection~\citep{deSouza2019}. \citet{deSouza2018} has said that the \HiSep~pressure and temperature operational conditions range are: 80-300 bar and 4-80 \textsuperscript{$\circ$}C. Moreover, \citet{deSouza2018} have presented three arrangements suggested by \citet{Passarelli2017} to improve separation: A, where the \CO~rich phase goes to a second gravitational separator to refine dense and oil phase separation; B, where other gravitational separator receives the \CO~poor phase to refine separation; C, a combination of the other two arrangments.  

\citet{Passarelli2019} have performed experiments mimicking the original composition and properties in the \HiSep~inlet stream. They~\citep{Passarelli2019} have tested 38\% \CO and GOR of 420 Sm\textsuperscript{3}/Sm\textsuperscript{3} for different temperature and pressure separation conditions, as presented in \tabref{tab:HiSepCondittionsExperiments}, to assess oil recovery yields. The minimun oil recovery yield reported was approximately 95\% for 250 bar and 60 \textsuperscript{$\circ$}C~\citep{Passarelli2019} . The oil recovery decreases with the increase in pressure and with the increase of the \CO~content due to the increase of the solubility of the light oil components in the dense gas phase~\citep{Passarelli2019}. In addition, the recovered oil GOR decreases steeply to a value around 120-130 Sm\textsuperscript{3}/Sm\textsuperscript{3}, considering that the inlet recombined oil GOR is 420 Sm\textsuperscript{3}/Sm\textsuperscript{3}~\citep{Passarelli2019}.
\begin{table}[H]
\captionsetup{labelfont={color=deepSeedGray},font={color=deepSeedGray}}
\caption{\HiSep~pressure and temperature tested by \citet{Passarelli2019} for 38\% \CO and GOR of 420 Sm\textsuperscript{3}/Sm\textsuperscript{3}}
\centering
 \begin{tabular}{>{\color{deepSeedGray}}c>{\color{deepSeedGray}}c}
 \hline
  Pressure (bar) & Temperature (\textsuperscript{$\circ$}C)\\
  \hline
  \multirow{2}{*}{150}& 60\\
    & 70\\
      \multirow{2}{*}{180}& 60\\
    & 70\\
    200 & 60\\
      \multirow{2}{*}{250}& 50\\
    & 60\\
    \hline
 \end{tabular}
\label{tab:HiSepCondittionsExperiments}
\end{table}

\citet{deSouza2018} has computed whole \HiSep~system using EMSO simulator. He~\citep{deSouza2018} has reported the P-T region where two phases coexist, a \CO~rich and an oily (liquid or vapor), for \CO~molar concentratios of 30, 50, 70 and 85\%. As a general trend, keep temperature fixed and increase the pressure has the effect to generate just one-phase. The higher the \CO~molar, the higher is this pressure. For temperature equal to 60\textsuperscript{$\circ$}, the oily phase is vapor (not a liquid) for any pressure/\CO-content values. Furthermore, the presence of light hydrocarbons (CH\textsubscript{4}) elevates the necessary pressure which two-phases equilibrium collapses in one~\citep{deSouza2018}. 

\citet{deSouza2018} has reported \citet{Passarelli2017}' proposed application process example. It contains a field with 30 to 90\% \CO-mol fraction with a temperature of 30\textsuperscript{$\circ$} and pressure of 120 bar ate gravitational separator inlet. In these condistions, \CO~rich phase represents from 70 to 95\% of mixture total volume and 70 to 90\% molar fraction of \CO whereas the oily phase from 5 to 30\% of total volume and 30 to 60\% molar fraction of \CO.

\citet{deSouza2019} have presented results with EMSO for gravitational separator at 100 bar and 57.1 \textsuperscript{$\circ$}C, necessary conditions to have 50\% of \CO~rich phase. The computated process was designed for 150$\times$10\textsuperscript{3} bpd of reservoir fluid~\citep{deSouza2019}. In reservoir conditions, \CO~content is 75\% with 89\% in \CO~rich phase and 68\% in oily phase~\citep{deSouza2019}. After the gravity separator, \CO~rich stream has 92\% in \CO~molar fraction while oily stream has 58\%~\citep{deSouza2019}. In order to decrease \CO~content in oily stream, it is possible to raise gravity separator temperature or lower its pressure~\citep{deSouza2019}. Despite oily phase still has high \CO~content, the \HiSep~decreases the \CO~total volume sent to topside treatment~\citep{deSouza2018}.

The power demand of the \HiSep~ system was 10.3 MW what could become the project economically unfeasible~\citep{deSouza2018}

\subsection{Membranes \CO-hydrocarbons separation}
\label{sec:MembranesCo2Hydrocarbons}
An alternative that has been studied, but it is not mature to be tested in a real field is the \CO-hydrocarbons separation using membranes. 

In 2019, a a joint industry project on subsea gas separation was launched with the following companies: Aker Solutions, Total, Pertamina, Equinor industry group the \CO~Capture Project (CCP) which includes BP, Chevron, and Petrobras as members~\citep{Berge2019}. According to \citet{Berge2019}, a prerequisite for the concept to be technically and economically attractive is that the gas separation is done with robust membranes that reduce pretreatment requirements and remove the need for large processing plants. Furthermore, it is necessary to test membranes under seabed conditions~\citep{Berge2019}.

\citet{Ommedal2021} has discussed about advantages and challenges to use membranes for \CO~subsea separation applications. His~\citep{Ommedal2021} work has focused on membranes of gas permeate class and he has presented the usual classfication of membranes used to gas natural separation. Moreover, \citet{Ommedal2021} has listed gas membrane separation challenges:
\begin{itemize}
 \item plasticization, membrane degradation due to swelling of porous material~\citep{Ommedal2021};
 \item membrane fouling, which is more common in separation processes with microfiltration, nanofiltration and reverse osmosis compared to gas separation~\citep{Ommedal2021};
 \item membrane pretreatment, due to particles, fouling and condensation of heavy hydrocarbons;
 \item suppliers of gas separation membranes;
 \item suited membranes material;
 \item pressure drop. 
\end{itemize}

\citet{Dalane2017} have stress out that despite membrane separation advantages such as small footprint, low capital and operation cost, high energy efficiency, great flexibility, simple to operate, no moving parts, low enviromental impact and ease to scale up, there are critical requirements in the selection of membrane modules for reliable performance and long term stability. Furthermore, the pre-treatment of the feed is not preferred since could introduce other complicated processes, or make it impossible to implement subsea~\citep{Dalane2017}.

\citet{Dalane2017} have also reported that membrane systems work quite effectively in small fields with low content of \CO~as Qadirpur whereas for bigger fields with high \CO~content (Indonesia, Thailand and Malaysia), the membrane treated streams still have a high \CO content and need further treatment.

In a recent work review about use of membranes to separate \CO-CH\textsubscript{4} in harsh conditions (pressure higher than 80 bar, low temperatures in depths above 2000 m), \citet{Cardoso2022} have enhanced that membrane-based separation is very used in offshore units to \CO~removal from natural gas. Nevertheless, traditional polymeric membranes have drawbacks related to plasticization and physical aging in subsea conditions~\citep{Cardoso2022}. Besides be capable to operate over a wide range of pressure and temperature, as natural gas feed can change composition over the years, the membranes used in \CO-CH\textsubscript{4} subsea separation, they have to deal  with multicomponent feed compositions similar to those found in raw natural gas since impurities present in natural gas can influence the efficiency of the membranes and compromise their selectivity and integrity~\citep{Cardoso2022}.

\newpage
